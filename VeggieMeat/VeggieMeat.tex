\documentclass[preprint, 3p,
authoryear]{elsarticle} %review=doublespace preprint=single 5p=2 column
%%% Begin My package additions %%%%%%%%%%%%%%%%%%%

\usepackage[hyphens]{url}

  \journal{An awesome journal} % Sets Journal name

\usepackage{lineno} % add
  \linenumbers % turns line numbering on

\usepackage{graphicx}
%%%%%%%%%%%%%%%% end my additions to header

\usepackage[T1]{fontenc}
\usepackage{lmodern}
\usepackage{amssymb,amsmath}
\usepackage{ifxetex,ifluatex}
\usepackage{fixltx2e} % provides \textsubscript
% use upquote if available, for straight quotes in verbatim environments
\IfFileExists{upquote.sty}{\usepackage{upquote}}{}
\ifnum 0\ifxetex 1\fi\ifluatex 1\fi=0 % if pdftex
  \usepackage[utf8]{inputenc}
\else % if luatex or xelatex
  \usepackage{fontspec}
  \ifxetex
    \usepackage{xltxtra,xunicode}
  \fi
  \defaultfontfeatures{Mapping=tex-text,Scale=MatchLowercase}
  \newcommand{\euro}{€}
\fi
% use microtype if available
\IfFileExists{microtype.sty}{\usepackage{microtype}}{}
\usepackage[]{natbib}
\bibliographystyle{plainnat}

\ifxetex
  \usepackage[setpagesize=false, % page size defined by xetex
              unicode=false, % unicode breaks when used with xetex
              xetex]{hyperref}
\else
  \usepackage[unicode=true]{hyperref}
\fi
\hypersetup{breaklinks=true,
            bookmarks=true,
            pdfauthor={},
            pdftitle={Plant-based meat alternatives and their associated microbial communities},
            colorlinks=false,
            urlcolor=blue,
            linkcolor=magenta,
            pdfborder={0 0 0}}

\setcounter{secnumdepth}{5}
% Pandoc toggle for numbering sections (defaults to be off)


% tightlist command for lists without linebreak
\providecommand{\tightlist}{%
  \setlength{\itemsep}{0pt}\setlength{\parskip}{0pt}}



\usepackage{setspace}
\doublespacing
\usepackage{caption}



\begin{document}


\begin{frontmatter}

  \title{Plant-based meat alternatives and their associated microbial
communities}
    \author[LMM]{Franz-Ferdinand Roch%
  %
  }
  
    \author[LMM]{Monika Dzieciol%
  %
  }
  
    \author[LMM]{Narciso Martin Quijada%
  %
  }
  
    \author[LMM]{Patrick-Julian Mester%
  %
  }
  
    \author[LMM]{Narciso Martin Quijada%
  %
  }
  
    \author[LMM]{Evelyne Selberherr%
  \corref{cor1}%
  }
   \ead{evelyne.selberherr@vetmeduni.ac.at} 
      \affiliation[LMM]{Unit of Food Microbiology, Institute of Food
Safety, Food Technology and Veterinary Public Health, Department for
Farm Animals and Veterinary Public Health, University of Veterinary
Medicine Vienna}
    \cortext[cor1]{Corresponding author}
  
  \begin{abstract}
  This is the abstract. It consists of two paragraphs.
  \end{abstract}
    \begin{keyword}
    keyword1 \sep 
    keyword2
  \end{keyword}
  
 \end{frontmatter}

\hypertarget{introduction}{%
\section{Introduction:}\label{introduction}}

For most of the population of the Western world, meat consumption is an
integral part of their diet. An average U.S. American consumed 100kg and
a European Union citizen 69kg of meat in 2021. The global meat
consumption increased from 21kg per year and capita in 1990 to 34kg in
2021 (OECD data). Although the OECD estimates that consumption will
level off at around 35kg per year and capita by 2030, the total meat
consumption will further increase with population growth. This globally
growing appetite for meat is linked to livestock farming and
consequently plays a major role in the ecological issues we are
currently facing. Land degradation, climate change, water pollution and
biodiversity loss are just some of the notable consequences for the
environment \citep{Steinfeld.2006, Bianchi.2018}. Additionally, common
industrial animal husbandry influences public health by supporting the
spread of antibiotic resistances and vector-borne disease
\citep{Economou.2015, Bianchi.2018, Watts.2018}. Apart from this, high
meat consumption, as practiced in the Western world and increasingly in
transition countries, contributes significantly to many widespread
common diseases, which, in addition to individual illnesses, also burden
health care systems
\citep{Micha.2010, Chan.2011, Parkin.2011, Feskens.2013}. Animal
welfare, environmental issues and increased health awareness are main
drivers for more and more people in western civilization to change their
meat consumption routines (\citet{StollKleemann.2017};
\citet{Ploll.2020}{]}. In a survey, conducted 2021 in ten different
European countries, 2\% of the participants referred to themselves as
vegan, 5\% as vegetarians, 3\% as pescetarians and 30\% as flexitarians
\citep{EuropeanUnionsHorizon2020reasearchandinnovationprogramme.2021b}.
The last group is characterized, among other things, by the fact that
they want to reduce their meat consumption, but do not want to give up
the positive experience that comes with it. As main target group,
flexitarians account for 90-95\% of the sales of vegan and vegetarian
products. A market research within ``The Smart Protein Project'' noted a
sales value increase of 82\% for plant-based meat (vegan and vegetarian)
within 2018 and 2020 for Austria
\citep{EuropeanUnionsHorizon2020reasearchandinnovationprogramme.2021}.
The sales of this product group and the number of different products
have increased strongly over the last few years
\citep{Curtain.2019, EuropeanUnionsHorizon2020reasearchandinnovationprogramme.2021}.
Despite the increased consumption of these products, there have been
only a few studies on their microbiological properties. However, these
would be important for ensuring food safety, characterizing potential
hazards, assessing risks, and for sustainability questions. Since two of
the UN's sustainability goals (goal 2 -- zero hunger and 12 --
responsible consumptions and production) affects our eating habits,
increased attention should be paid to reducing food waste. As 30\% of
food products in primary processing does not even reach the consumer,
mainly because of microbial spoilage or pathogen contamination, it is
essential to improve the knowledge of the microbial communities of our
food, which could help to increase the shelf life and reduce the
contamination with pathogens. Still largely unanswered is the question
if and how a high microbial diversity on food (consisting of living and
dead microbiota) can have a positive effect on consumer's health. For
example, a high microbial diversity permanently stimulates the innate
and adaptive immune system and provides resistance against colonizing
pathogens \citep[\citet{Mackowiak.1982},
\citet{Smith.2007}]{Crowe.1973}. It was also supposed recently, that the
loss of microbial diversity including the disappearance of ancestral
indigenous microbiota, which is currently happening in western
countries, affects human health and contributes to post-modern
conditions such as obesity and asthma \citep{Blaser.2009, Vangay.2018}.
For all these research questions fundamental knowledge on the microbial
compositions on food are necessary, but still lacking. Since, little is
known on these products, we sampled a selection of plant-based meat
alternatives available in Austria's supermarkets to investigate the
general microbial community patterns. Further we described
characteristic microbial profiles and compared four groups of the most
common product types (pea and soybean based products with either
``minced'' (minced meat, burgers, etc.) or ``fibrous'' (meat chunks,
Schnitzel, etc.) texture. We hypothesized that products within one group
have, based on similar protein processing, more similar communities than
between the groups. Since we examined a relatively undescribed product
group, we have chosen a combined approach using culture-dependent and
culture-independent techniques. This allowed a more complete description
of the microbial community than any of the applications could stand on
their own, but also raised additional questions and uncertainties that
are relevant to future microbial research of (highly processed) foods.

\hypertarget{material-and-methods}{%
\section{Material and Methods}\label{material-and-methods}}

\hypertarget{sample-acquisition}{%
\subsection{Sample acquisition}\label{sample-acquisition}}

We purchased 32 different plant-based meat alternative products, between
July 12 and July 14, 2021, from four supermarket groups. Pea- and
soybean- protein based products with either a minced or a fibrous
structure were the most common representatives of this product group at
this time point in Austria. Additional criteria for the selection were,
that the products where entirely plant-based (vegan) and do not contain
fermented products, like tofu. Beside these characteristics, the samples
were different in their composition, packing, shelf-life etc. (Table 1).
All samples, including two frozen products, were transported
refrigerated and stored at 4°C until processing. Sample processing took
place within the minimum shelf life range and latest five days after
purchase.

\hypertarget{sample-preparation}{%
\subsection{Sample preparation}\label{sample-preparation}}

In total, 10 g of each sample, representing all layers of a product,
were placed in sterile Stomacher bags and diluted with 90 ml sterile
PBS. A Stomacher \ldots{} was used to mechanically comminute the samples
for 120 s. To remove coarse food particles, 45 ml of the homogenate was
centrifuged at 300 relative centrifugal force(rcf) for 2 min at room
temperature using an Eppendorf Centrifuge 5810R and an A-4-62 rotor. The
remaining supernatants were transferred to new tubes and centrifuged at
3,000 rcf (30 min at RT). The obtained cell pellets were diluted
10-times with sterile phosphate buffered saline (PBS).

\hypertarget{bacterial-and-fungal-isolation}{%
\subsection{Bacterial and fungal
isolation}\label{bacterial-and-fungal-isolation}}

We carried out two sets of cultivation experiments. The first set was
more general with a basic set of non-selective and selective media
(Columbia agar, Violet Red Bile Dextrose agar (VRDB), brain heart
infusion agar (BHI), plate count agar (PCA), Rose Bengal (RB), and Baird
Parker (BP) agar), to get a broad range of the expected microbial
community. The plates were inoculated with a 102 dilution and
aerobically and semi-anaerobically incubated at 37°C. In order to
recover high amounts of different isolates, the plates where incubated
for 16-68 hours, depending on colony size and growth density. Samples
with too high growth densities after 16 h for picking single colonies
were diluted 103 or 104 in sterile PBS and plated again. For the second
media set we used the information from the 16s rRNA ampicon sequencing
to select specific media and growth conditions in order to isolate
representatives of genera we could not isolate under the first set's
conditions. Depending on the sample specific microbial composition we
used Luria Bertani-, nutrient-, tryptic soy-, marine-, corynebacteria-,
De-Man-Rogosa-Sharpe- and pseudomonas agar and cultivated the samples
(dilution 102-105) anaerobically at 25°C for 48h. In both sets we
selected morphologically unique, single colonies for re-cultivation and
Sanger sequencing. DNA was extracted, using a protocol modified after
Walsh et al. \citep{Walsh.2013}, by lysing pure cultures with
100\(\mu\)l 0.01M Tris-HCl and 400\(\mu\)l 2.5\% Chelex 100 resin
solution (BioRad) at 95°C for 10min, followed by centrifugation with
15,000 rcf for 30 sec.~The supernatants were subsequently used for 16S
rRNA gene PCR, using a final concentration of 200nM of each of the
universal Primers from LGC (27F -- 5'-GAG TTT GAT CMT GGC TCA G-3' and
1492R -- 5'-GGY TAC CTT GTT ACG ACT T-3'), 0.025 U/\(\mu\)l
Invitrogen\textsuperscript{TM} Platinum\textsuperscript{TM} Taq
DNA-Polymerase, 1x TaqMan PCR buffer, 2mM MgCl2, and 250nM dNTP Mix
(Thermo Scientific\textsuperscript{TM}). For the reaction a protocol of
95°C for 5 min (Taq activation) followed by 35 cycles of 40 seconds at
95°C (denaturation) 40 seconds at 52°C (annealing ) and 1 minute at 72°C
(elongation) was used. For negative control a reaction with ddH20 as
well as a reaction the negative extraction control were performed.
In-house Listeria monocytogenes DNA served as positive control. All PCR
products were checked with a QIAxcel DNA High Resolution Kit in a
QIAxcel Advanced system. Negative PCR results were followed by an ITS2
gene PCR (200nM of each of the primers ITS3-5'-GCATCGATGAAGAACGCAGC-3'
and ITS4-5'-TCCTCCGCTTATTGATATGC-3' (White et al.~1990), 0.025
U/\(\mu\)l Invitrogen\textsuperscript{TM} Platinum\textsuperscript{TM}
Taq DNA-Polymerase, 1x TaqMan PCR buffer, 2mM MgCl2, and 250nM dNTP Mix
(Thermo Scientific\textsuperscript{TM}), with a protocol of 95°C for 5
min (Taq activation) followed by 30 cycles of 40 seconds at 94°C, 40
seconds at 56°C and 1 minute at 72°C). For cultures, negative in 16S
rRNA and ITS2 PCR, we repeated the extraction with the NucleoSpin tissue
kit (Machery-Nagel), using the manual in combination with the
recommendations for hard to lyse bacteria. LGC Genomics GmBH (Berlin)
purified and sequenced the PCR products in one direction (using 27F
primer for 16S rRNA and ITS4 for the ITS2 region). For quality trimming
we used ``method 2'' (with a defined sliding window of 15 bp and an
Phred score mean quality of 40) within the ``SangerRead'' function
(``sangeranalyseR'' package; R {[}\citet{R} Core Team 2021)). Trimmed
sequences, with \textgreater100 bp length, were classified with the
``assignTaxonomy'' function of the ``dada2'' package in R (with kmer
size 8 and 50 bootstrap replicates), based on a RDP Naïve Bayesian
Classifier algorithm (Wang et al.~2007). Potential pathogens and
unclassified Enterobactericeae were further whole genome sequenced on a
MinIon MK\ldots{}

\hypertarget{bacterial-quantification-and-composition}{%
\subsection{Bacterial quantification and
composition}\label{bacterial-quantification-and-composition}}

For direct DNA extraction the DNeasy PowerFood kit was used. That for,
the in RNAlater stored cell pellets were thawed, centrifuged (3,000×g,
30min) and resuspended in ??? \(\mu\)l MBL buffer. Deviating from the
DNeasy PowerFood protocol, the lysis step was proceeded in \ldots{}
tubes. The further steps were processed according to the protocol. The
elution step was done twice with 30\(\mu\)l sterile water, respectively.
Parallel an extraction of pure water was performed as negative
extraction control (NEC) Sequencing libraries of the 16S rNRA gene (V3/4
region) were prepared based on Illumina 16S Metagenomic Sequencing
Library Preparation recommendations. Primers 341F
(5′-CCTACGGGNGGCWGCAG-3′) and 805R (5′-GAC TAC HVG GGT ATC TAA TCC-3′)
(Klindworth et al.~2013) were used together with Illumina adapter
sequences (5′ CGT CGG CAG CGT CAG ATG TGT ATA AGA GAC AG-3′ and 5′ GTC
TCG TGG GCT CGG AGA TGT GTA TAA GAG ACA G-3′, respectively) for
amplification. Libraries were constructed by ligating sequencing
adapters and indices onto purified PCR products using the Nextera XT
Sample Preparation Kit (Illumina). Equimolar amounts of each of the
purified amplicons were pooled and sequenced on an Illumina MiSeq
Sequencer with a 300-bp paired-end read protocol. 16S rRNA gene amplicon
library generation and sequencing was performed at the Vienna Biocenter
Core Facilities NGS Unit (www.vbcf.ac.at).

\hypertarget{sequence-processing-and-statistics}{%
\subsection{Sequence Processing and
Statistics}\label{sequence-processing-and-statistics}}

The data obtained from the MiSeq sequencing went through a Qiime2
workflow including demultiplexing and DADA2 denoising and QC filtering.
The taxonomic classification was done with the Scikit-learn algorithm
using a pre-trained full-length-uniform-classifier based on the SILVA
138.1 database. After removal of sequences with mitochondrial or
chloroplastic origin, further analyses were performed with R. To
estimate the alpha-diversities of the samples Hill-Simpson and
Hill-Shannon diversity were calculated using the ``iNEXT'' package,
based on 99.5\% coverage rarefied samples (Roswell et al.~2021). Group
comparisons were done with Kruskal-Wallis tests followed by
Bonferroni-alpha-corrected Dunn's tests for pairwise comparisons. For
beta diversity analyses the samples were rarefied with 100 iterations
based on a coverage of 99.5\% using the package ``phyloseq'' and
``metagMisc'' {[}\citet{Chao} und Jost 2012). Based on that we generated
distance matrices with Bray-Curtis dissimilarities, Jaccard indices and
Jensen-Shannon divergence and use them for graphical (t-distributed
stochastic neighbor embedding - tSNE) and statistical (PERMANOVA, LEfse)
analysis. For tSNE we used the ``Rtsne'' package with a maximum of 1000
iterations, a perplexity of 5 and two initial dimensions, as recommended
by Oskolkov {[}\citet{Oskolkov} 2019). The high variability of the
products and little knowledge on the underlying production conditions
made it impossible to find adequate variables for PERMANOVA, but the
main protein source and the texture. The producing facility as
additional variable would be meaningful, but the product assortment is
dominated by one company, which made the model design very unbalanced.
PERMANOVA was done with the ``vegan'' package using ``betadisp'' to
check for homogeneous dispersion and ``adonis'' functions for PERMANOVA
with 999 iterations using main protein source and texture as explanation
variables. Lefse was done with the relative abundance data of the
coverage rarefied data in combination with ``phyloseqCompanion'' R
package, for data transformation and the Lefse Bioconda tool by \ldots,
using the protein source and texture as class, a normalization of 1
million and a log\textsubscript{10} LDA score threshold of 4.0. In
parallel, a group comparison for the same features, as examined with
Lefse, was done with Kruskal-Wallis tests followed by a
Benjamin-Hochberg alpha correction. The Sanger sequences from the
Isolates were trimmed using the ``sangeranalyseR'' package with a
Quality Score Cutoff of 40 and a sliding window size of 15. Sequences
with a minimum length of 100bp and a trimmed mean quality score
\textgreater35 were used for taxonomic classification based on the SILVA
138.1 database. Further, the isolate sequences were assigned to a
database generated from the MiSeq data to connect the culture-based and
culture-independent approaches. For figure 1, the isolate sequences of
each genus were clustered within each sample, group and producer using
the IdClusters function from the ``DECIPHER'' package with a cutoff of
0.06. More clusters within each genus were interpreted as a higher
species or strain diversity within each genus. Selected isolates were
whole genome sequenced with the MinIon, using\ldots{} The obtained data
were trimmed and filtered with filtlong, assembled with Flye followed by
several polishing steps (four repetitions of racon and a final step with
medaka), before they were used in the TORMES workflow.

\hypertarget{results}{%
\section{Results}\label{results}}

\hypertarget{product-descriptions}{%
\subsection{Product descriptions}\label{product-descriptions}}

The four groups did not have the same sizes, since the product portfolio
for each group was not large enough. Still, we had an overall of 16
pea-based and 16 soybean-based products (Table 1). Beside the four main
properties, the products were quite diverse. Particularly noteworthy is
the number of ingredients used per product (4-27, with a total of about
120 different used ingredients within the 32 sampled products) and the
large range of shelf life. The products are also different on factors,
which might influence the bacterial composition, i.e.~pre-heating or
freezing steps, as well as packing in modified atmosphere. Most of the
products had clear cooking instructions on the labels, including the
recommendation for thorough cooking (Table 1). In total, 27 samples were
packed in modified atmosphere with unknown composition. Out of 30
samples sold refrigerated (the other two frozen) six products were
frozen at any point during the retail chain (Table 1).

\hypertarget{cultivable-microbial-communities}{%
\subsection{Cultivable microbial
communities}\label{cultivable-microbial-communities}}

In total, 465 colonies were picked and selected for 16S rRNA or ITS gene
sequencing. Among these, 431 could be classified to the genus level,
representing 38 genera in four different phyla (Figure 1). The remaining
isolates could only be assigned to family level (n=16, all
Enterobactericeae), were fungi (n=15; Wickerhamomyces (n=7),
Issatchenkia (n=6), Yarrowia (n=1), Dipodascus (n=1)), or stayed
unassigned (n=3). We isolated Bacillus from 19, Leuconostoc from 18,
Enterococcus from 12, Latilactobacillus from 10 samples. Species from
the genus \emph{Bacillus}, Leuconostoc and Latilacto\emph{Bacillus}
could be isolated from each of the four sample groups.
\emph{Enterobacteriaceae}, which are usually surveyed as an additional
hygiene criterion, were only found in the pea protein products of a
single manufacturer (Figure 1). A selection of these
\emph{Enterobacteriaceae} and isolates classified as potential pathogens
(i.e.~\emph{Staphylococcus aureus}, \emph{Bacillus cereus} group,
\emph{Klebsiella} sp.) were whole genome sequenced with a MinION device.
DATA MISSING

\hypertarget{lactic-acid-bacteria-and-gamma-proteobacteria-dominate-the-16s-rrna-amplicon-sequences}{%
\subsection{\texorpdfstring{Lactic acid bacteria and
gamma-\emph{Proteobacteria} dominate the 16S rRNA amplicon
sequences}{Lactic acid bacteria and gamma-Proteobacteria dominate the 16S rRNA amplicon sequences}}\label{lactic-acid-bacteria-and-gamma-proteobacteria-dominate-the-16s-rrna-amplicon-sequences}}

In total, 28 samples (883,866 sequences; median frequency per sample:
25,627; range: 439-253,681) passed the quality criteria and were
processed with QIIME 2. Because we used coverage based rarefaction (with
a coverage of 99.5) and all of the remaining samples meat this coverage,
we removed none of them for further analysis. Over all samples, the ASVs
were assigned to 25 different Phyla, however, only three Phyla with
\textgreater3\% in at least one of the samples were found
(i.e.~\emph{Firmicutes} 0.00-0.95\%, \emph{Proteobacteria} 0.00-0.31\%,
\emph{Bacteroidota} 0.00-0.11\%). In total, 18 samples were dominated
(\textgreater50\% relative abundance) by \emph{Firmicutes} (10 samples
with \textgreater90\%), while in the other 10 samples
\emph{Proteobacteria} is the most abundant Phylum. The most common
genera were \emph{Leuconostoc} (detected in 26 samples; 0.03-100.00\%
rel. abundance), \emph{Latilactobacillus} (detected in 21 samples;
0.02-86.38\% rel. abundance), \emph{Pseudomonas} (detected in 21
samples; 0.36-35.25\% rel. abundance), \emph{Serratia} (detected in 19
samples; 0.03-8.92\% rel. abundance), and Acinetobacter (detected in 18
samples; 0.09-15.40\% rel. abundance). The genus Leuconostoc was the
most abundant in 13 samples, followed by Latilactobacillus (4 products),
and Shewanella (4 products) (Figure 2). Some genera were found
proportionally high (≥10\%) in one or more samples, but could not be
isolated (i.e.~\emph{Shewanella}, \emph{Xanthomonas},
\emph{Photobacterium}, \emph{Myroides}, \emph{Pediococcus}).

\hypertarget{protein-source-and-texture-are-not-the-main-driver-for-the-community-pattern}{%
\subsection{Protein source and texture are not the main driver for the
community
pattern}\label{protein-source-and-texture-are-not-the-main-driver-for-the-community-pattern}}

The Kruskal-Wallis tests comparing the alpha-diversity indices
(i.e.~Hill-Shannon index and Hill-Simpson index) between the four groups
(based on proteins source and texture) were significant (p.value = 0.018
and 0.049), but in post-hoc Dunn's test with Bonferroni alpha adjustment
only the Hill-Shannon index between groups ``pea-fibrous'' and
``pea-minced'' differed significantly (p.value= 0.015 - Supplement
Figure xy). The group dispersions were homogenous in all examined
distance methods (i.e.~Bray-Curtis, Jaccard, JSD). The PERMANOVA showed
that texture and protein source significantly affects the microbial
composition (Supplement Table 2), but explained only between
15.4629221and 22.7898534\% of the total variance. The variance
explanation by the PERMANOVA would increase, if the manufacturer as
variable was added to the model, but since the sampling was very
unbalanced on that, we avoided this step.

LEFSE FEHLT HIER

\hypertarget{s-rrna-sequencing-revealed-three-distinct-community-profiles}{%
\subsection{16S rRNA sequencing revealed three distinct community
profiles}\label{s-rrna-sequencing-revealed-three-distinct-community-profiles}}

In the tSNE plot we see three distinct clusters (Figure 3), which we
described, based on the dominating genera as \emph{Leuconostocaceae}-,
\emph{Latilactobacillus}- and \emph{Proteobacteria}-profiles. The
clustering is traceable, when comparing the similarity of the relative
abundance patterns of the samples within each cluster. underlined the
heterogenic microbial pattern of Figure 2. Although there is some
clustering, there is no clear separation based on the examined variables
(main protein source, status, manufacturer).

ALPHA DIV AND LEFSE FOR PROFILES LEfSe identified 32 discriminative
features with an LDA Effect Size \textgreater4.0 (Fig 3). This analysis
highlighted a predominance of Leuconostocaceae (log LDA 5.48) and its
classification levels above in pre-cooked pea products. It identified
\emph{Proteobacteria} (log LDA 5.33) as most characteristic for raw pea
products, mainly based on the predominance of Pseudomonas (log LDA
4.81). Highly discriminant for soy products were some low abundant
features. LEfSe calculated and log LDA score of 4.68 for Acetobacter in
pre-cooked soy products and of 4.62 for Flavobacteriaceae in raw soy
products.

\hypertarget{discussion}{%
\section{Discussion}\label{discussion}}

Highly processed food, like the plant-based meat alternatives we
examined in this study, brings challenges to microbiologists. The broad
range of ingredients (\textasciitilde120 in in the 32 examined
products),different processing steps, and a variety of equipment lead to
many potential sources for bacterial contamination. The product specific
production processes were not available, but up to the present low and
high moisture extrusion are most common commercially used technologies
to produce meat textured plant proteins. Both methods are based on an
interaction of heat, shear forces and pressure. The conditions for this
process depends on the original protein source and the desired protein
structure. During the extrusion, the protein dough undergoes temperature
gradients between starting at room temperature and ending between 130
and 160°C at a pressure of\ldots. Blabla et al., described, that the
dough remains about two minutes at the final high temperatures.
Currently, we assume, that only spore forming bacteria like
\emph{Bacillus} would survive the extrusion process. That is also
described by Domig et al., who could only isolate \emph{Bacillus} spp.
from extruded proteins\ldots(i.e.~\emph{Bacillus} and
\emph{Clostridium}) can survive \citep{Leutgeb.2017}. This could be one
explanation why was isolated from most of the samples (19/32), while the
relative abundance of 16S rRNA gene DNA is relatively low, compared to
other genera. Overall, we assume, due to this initial extrusion
procedure, the main protein is not the source for most of the living
bacterial cells we have isolated from the products. The extrusion
process (and other heating and freezing steps during the production)
brought challenges for the interpretation of sequencing data. It is in
the principle of the method that it cannot differentiate between living
and dead, which raises the question of whether the use of 16S rRNA
metagenomics is an appropriate means of investigating highly processed
foods. Along with these processes, there are many changes in the
environmental conditions for the products and their associated microbial
communities, like heating, thawing, freezing and cooling steps, but also
packing in modified atmosphere. These conditional changes could kill or
inactivate bacterial cells, while their DNA remains largely intact, but
could also induce viable but nonculturable states
\citep{Li.2014, Zhao.2017}. We agree, that it should not be used as
standalone method, but it i) supports culture-based methods and allows
to adapt media and culture conditions to genera we find in the
sequencing data, which is especially important for less examined
environments (i.e.~new foods), ii) is able to detect viable but non
culturable bacteria, and iii) uncovers contaminations during the
process, which could help to adapt HACCP concepts. In this study we
examined nine samples with high relative abundancies of
\emph{Proteobacteria}, mainly \emph{Pseudomonas}, \emph{Psychrobacter}
and \emph{Shewanella}. These three genera, are common spoilers in meat
and fish products \citep{Odeyemi.2018}., especially in products with
slightly higher pH values, like fish, sea food and poultry. Plant-based
meat products are usually also slightly higher in pH (Quellen). Although
the number of ASVs within a these genera is much higher, compared
e.g.~with \emph{Leuconostoc} from other samples, what we associate with
different sources of contamination, there are still single ASVs with
relatively high abundancies, which indicates either a large entry from
one source or an active growth on the product at some time point during
the production. The most obvious sources would be the main protein, the
used water (via biofilms in water reservoirs or water hoses) or biofilms
during the process. From the found genera in the MiSeq data, at least 13
are described as biofilm builders in food processing environments
(Wagner et al.), among these \emph{Pseudomonas}, \emph{Psychrobacter}
and \emph{Shewanella}. The possibility of biofilm formation on the
processing equipment is high, since many of the used machines are hard
to clean. However, this is contradicted by the fact that among the nine
\emph{Proteobacteria}-dominated samples seven had a very similar
pattern, but were from six different producers. There are no available
studies on the microbial communities of raw soybean or pea proteins, but
16S rRNA patterns of peas or the phyllosphere of soy do not support the
thesis, that this kind of contamination is associated with the main
protein source. Only sample B4, which is the only sample, that is
dominated by \emph{Alphaproteobacteria}, had high similarities to the
microbial community of the soy phyllosphere \citep{Vorholt.2012}.
However, some of the producer just have a few products in this food
segment, so we would not exclude, that they buy already extruded
proteins from a large distributor and use it for their products.\\
The comparison of the diversity of the different groups is limited.
Since the study design was focused on describing the microbiota of
different meat alternatives and to detect possible differences between
main protein sources and texture, it was not quite balanced in other
perspective. The two most frequently represented manufacturers
contributed 18 (11+7) out of 32 products in this study, while the other
seven manufacturers contributed with a maximum of two samples. As shown
in Supplements Table 1 manufacturers have usually specialized their
products to one of the four examined groups. We assume that the
production plant has a non-negligible effect on the product's
microbiota, which could be seen in part in Figure 3. Based on the
microbial distribution patterns from the MiSeq data, we roughly describe
three different community compositions. The \emph{Leuconostocaceae}
dominated samples (14/27), the \emph{Latilactobacillus} dominated
samples (4/27) and the \emph{Proteobacteria} dominated samples. In the
first two groups each sample is clearly dominated by one ASV
respectively. That speaks for the fact, that there was an active and
growing population at any timepoint on the product. Beside of three
samples, the dominating ASV assignable to an isolate, therefor we
assume, that these species are actually active in the final product.
This is especially true for the Lactic acid bacteria. The role of the
isolated and sequenced LABs (especially \emph{Leuconostoc mesenteroides}
and \emph{Latilactobacillus sakei}) on this product group has to be
assessed from several perspectives. They are fermenters on many food
products and contribute there to the formation of substances relevant to
taste. They inhibit with their metabolites (i.e.~organic acids) and
their secondary metabolites (i.e.~Bacteriocin) the growth of pathogens
and spoilage bacteria, but they are spoilage bacteria themselves,
especially on meat and meat products. Most of the isolated LABs have
ambiguous positions in foods and food production. While
\emph{Latilactobacillus} and \emph{Leuconostoc} were used for
fermentation of tofu, tempeh or Kimchi, they were also commonly
described as food spoilers in non-fermented products, like fresh meat
\citep{Casaburi.2015}. Since the examined products were neither
assignable as fermented food, nor had fermented products as ingredients
(besides vinegar), we associated the isolated genera primarily in their
position as product spoilers and not as part of a planned fermentation
process. In 18 out of 27 samples with utilizable MiSeq data were lactic
acid bacteria dominating (Leuconostoc 13, Lacto\emph{Bacillus} 4,
Weissella 1), which is expectable as LABs are described as dominating
species on fruits and vegetables {[}\citet{Yu} et al.~2020). While ten
of the Leuconostoc dominated products were pea protein based, all four
\emph{Latilactobacillus}-dominated samples were soy products. The
microbial community of all of the pea products with fibrous structure
had \textgreater95\% abundances of LABs (\emph{Leuconostoc},
\emph{Latilactobacillus}, \emph{Weissella}, \emph{Carnobacterium}).
\emph{Leuconostoc} is described to have a starting advantage over other
LABs on plant products, based on its ability to \ldots{} According to
that Leuconostoc is often described as dominating LAB at the beginning
of a fermentation process.\\
Diverse \emph{Latilactobacillus sakei} strains wurden erfolgreich auf
verschiedensten Produkten als Biokonservativ getestet und ist im
Regelfall im Vorteil gegenueber Leuconostoc. Zum einen spielen beide
Spezies eine Rolle in der Fermentation von diversen pflanzlichen
Produkten. In Kimchi ist Leuconostoc in den ersten Tagen, der
dominierende LAB und wird zunehmend von \emph{Latilactobacillus sakei}
verdraengt {[}\citet{Jeong} et al.~2013) Gaaerts et al., 2020 examined
ready to vegetarian, vegan and insect based meat alternative, including
products with fermented ingredients, like fermented sour dough. Based on
that, they assumed sour dough as main source for \emph{Latilactobacillus
sakei}, which they isolated most. Different to that study, we excluded
products with fermented ingredients, but vinegar. Still we found large
amounts of lactic acid bacteria in the examined products. Unlike
Graeerts et al., we isolated Leuconostoc mesenteroides from the majority
of the samples. This species is an obligate heterofermentative LAB,
using the phosphoketolase pathway for glycolysis, with lactic acid,
acetic acid, CO2 and ethanol as metabolites{[}\citet{Gaenzle} 2015).
\ldots{} While \emph{Bacillus} spp. were frequently described as
spoilers in bakery and dairy products, but also in meat and poultry,
Leuconostoc spp. were mainly associated with the spoilage of meat and
seafood but also of vegetables (minimally processed and ready to eat
products \citep{Lianou.2016}. Lacto\emph{Bacillus} (meat, seafood)
pseudomonas (meat, seafood, milk). Undoubtedly, some manufacturers are
guided by those similarities when determining their products shelf life
and consumption recommendations. Two manufacturer labelled all their
products with expiry dates, the others with best before dates. We think
that because of the novelty of these products and accompanying missing
experience, consumers might not be able to recognize spoiled products by
odor or taste. However, broader investigations tailored to this product
category and considering the manifold microbial sources, could increase
product stability, facilitate manufacturer's decision for product
labelling, and help consumer to recognize spoiled products. Altogether,
could reduce food waste and increase sustainability of these products.
Beside the dominating, spoilage associated genera, we alos isolated some
Enterobactericeae (i.e.~Raoultella, Kosakonia, Klebsiella,
Escherichia/Shigella, Enterobacter, Citrobacter, Atlantibacter). All of
these isolates came from products of the same manufacturer. In MiSeq the
relative abundances of \emph{Enterobacteriaceae} were very low in all
examined samples. All products were labelled with cooking instructions,
most of them specific with cooking time in minutes, others only with
``heat through before consumption''. This final heating step by the
consumer can be a part of the HACCP concept of the manufacturer. While
cooking time in minutes is a good guidance for the consumers, ``heat
through'' alone is in our opinion too less information. First there is a
lack of experience with these kind of products, second in contrast to
real meat (products) there is no indicator like color change for most of
these products to determine a sufficiently cooked state.
\emph{Enterobacteriaceae} widely serve as hygiene indicators. According
to the sequence data \emph{Enterobacteriaceae} had a minor role in the
examined samples. However, we isolated seven different genera of
Enterobactericeae, among these Enterobacter, Salmonella, Klebsiella and
Citrobacter. This indicates, that the conditions in this product group
are given for the survival of \emph{Enterobacteriaceae}, including
potential human pathogenic species like Enterobacter aerogenes,
Salmonella enterica, Klebsiella oxytoca and Citrobacter freundii.
Knowing this, we would recommend process hygiene criteria for this
product group and offering preparation instruction (incl.~cooking time)
for all of the products.\\
General statements about the microbiological risk of these products are
difficult to make, since producers, consumers, regulatory authorities,
and scientist are not quite experienced with these kind of products.
(gueltige regulierungen, haccp concept) In terms of product safety, four
critical issues are in the hands of consumers: spoilage detection,
handling of the products and used utensils before cooking, the
cooking/heating step and the storage for later consumption. Spoilage
detection: 2 producer with expiration dates (orientierung an
faschierten) rest laesst es den Konsumenten. Keine Erfahrungswerte
unmoeglich am geruch zu erkennen, jedes produkt andere rezeptur. Mhd
produkte tendetiell hoehere BCE. Große schwankungen in den shelf lifes.
Handling of the products: kaum Erfahrung. Wird es wie Fleisch behandelt
oder eher nachlaessig. Allgemeine Kuechenhygiene, eigene utensilien bzw.
wschschritte handhygiene etc. wir sehen das pathogen in den rohen
produkten wachsen koennen. Selbst wenn fleisch erhitzt wird bei
fehlender kuechenhygiene hab ich das dann halt wo anders Cooking/heating
step. Most oft he products have cooking instructions. Studie hat gezeigt
dass die standard pathogene absterben wenn man die produkte wie fleisch
entsprechend erhitzt. Fehlende instructions nicht befriedigend. Nur
durcherhitzt verzehren als instruction zu wenig, da aqnders als bei
Fleisch eine durcherhiztung optisch nicht erkannt werden kann Storage
for later consumption: studie der ungarn. Halten tendentiell kuerzer als
die fleisch pendants im kuehlschrank nach der zubereitung.

\hypertarget{conclusion}{%
\section{Conclusion}\label{conclusion}}

\hypertarget{conflict-of-interest}{%
\section{Conflict of Interest}\label{conflict-of-interest}}

The authors confirm that they have no conflicts of interest with respect
to the work described in this manuscript.

\hypertarget{acknowledgements}{%
\section{Acknowledgements}\label{acknowledgements}}

We would like to thank Birgit Bromberger and Christoph Eisenreich for
their active support in conducting the experiments.

\hypertarget{contributions-of-authors}{%
\section{Contributions of Authors}\label{contributions-of-authors}}

\hypertarget{tables}{%
\section{Tables}\label{tables}}

\hypertarget{figures}{%
\section{Figures}\label{figures}}

\hypertarget{supplements}{%
\section{Supplements}\label{supplements}}

\renewcommand\refname{References}
\bibliography{VeggieMeat.bib}


\end{document}
